\begin{center}
{\LARGE\bfseries COBOL Cheat Sheet}\\[0.5em]
{\large Hilfestellungen für Aufgabe 2 (a und b)}
\end{center}

\vspace{1em}

% ============================================================
% Aufgabe 2a – Einlesen, Aufbereiten, Zerlegen
% ============================================================

\begin{cobolCodeSection}{teal!60}{Leerzeilen behandeln und Tabs entfernen}
\begin{lstlisting}[language=Cobol]
  MOVE FUNCTION TRIM(BUCHUNG-LINE) TO LINE-TRIM.
  INSPECT LINE-TRIM REPLACING ALL X"09" BY " ".   *> Tabs -> Spaces

  IF LINE-TRIM = SPACES
     OR LENGTH OF FUNCTION TRIM(LINE-TRIM) = 0
     CONTINUE   *> Leere Zeile überspringen
  END-IF.
\end{lstlisting}
\end{cobolCodeSection}

\begin{cobolCodeSection}{teal!60}{Strings per UNSTRING zerlegen}
\begin{lstlisting}[language=Cobol]
  MOVE SPACES TO WS-VORNAME WS-NACHNAME
                 WS-STUNDEN WS-GEHALT WS-GEBURT.

  UNSTRING LINE-TRIM
      DELIMITED BY ALL SPACE
      INTO WS-VORNAME
           WS-NACHNAME
           WS-STUNDEN
           WS-GEHALT
           WS-GEBURT
  END-UNSTRING.
\end{lstlisting}
\end{cobolCodeSection}

\begin{cobolCodeSection}{teal!60}{Strings zusammenbauen (STRING)}
\begin{lstlisting}[language=Cobol]
  MOVE FUNCTION TRIM(WS-VORNAME) TO WS-NAME-ZUS.
  MOVE 1 TO WS-J.

  STRING
      " " DELIMITED BY SIZE
      FUNCTION TRIM(WS-NACHNAME) DELIMITED BY SIZE
      INTO WS-NAME-ZUS
      WITH POINTER WS-J
  END-STRING.
\end{lstlisting}
\end{cobolCodeSection}

\begin{cobolCodeSection}{teal!60}{Längenprüfung eines Strings}
\begin{lstlisting}[language=Cobol]
  MOVE LENGTH OF FUNCTION TRIM(WS-GEBURT) TO WS-LEN.

  IF WS-LEN NOT = 10
     *> Fehlerfall behandeln
  END-IF.
\end{lstlisting}
\end{cobolCodeSection}



% ============================================================
% Aufgabe 2b – Validierung & Fehlerprotokollierung
% ============================================================

\begin{cobolCodeSection}{teal!60}{Elektronische Zeichenprüfung (INSPECT TALLYING)}
\begin{lstlisting}[language=Cobol]
  MOVE 0 TO WS-CNT.

  INSPECT ALLOWED-NAME-CHARS
     TALLYING WS-CNT
     FOR ALL WS-NAME-ZUS(WS-I:1).

  IF WS-CNT = 0
     *> Sonderzeichen gefunden
  END-IF.
\end{lstlisting}
\end{cobolCodeSection}

\begin{cobolCodeSection}{teal!60}{Numerische Prüfung}
\begin{lstlisting}[language=Cobol]
  IF WS-STUNDEN(1:1) = "-"
     *> Negative Werte nicht zulässig
  END-IF.

  IF WS-WERT NUMERIC
     *> gültige Zahl
  ELSE
     *> Fehler: nicht numerisch
  END-IF.
\end{lstlisting}
\end{cobolCodeSection}

\begin{cobolCodeSection}{teal!60}{Mehrfach-Trenner prüfen}
\begin{lstlisting}[language=Cobol]
  MOVE 0 TO WS-CNT.
  INSPECT WS-STUNDEN TALLYING WS-CNT FOR ALL ".".

  IF WS-CNT > 1
     *> zu viele Dezimalpunkte
  END-IF.
\end{lstlisting}
\end{cobolCodeSection}

\begin{cobolCodeSection}{teal!60}{Datum zerlegen und prüfen}
\begin{lstlisting}[language=Cobol]
  MOVE WS-GEBURT(1:2) TO WS-TAG.
  MOVE WS-GEBURT(4:2) TO WS-MONAT.
  MOVE WS-GEBURT(7:4) TO WS-JAHR.

  IF WS-MONAT < 1 OR WS-MONAT > 12
     *> Ungültiger Monat
  END-IF.

  *> Max. Tageszahl je Monat (inkl. Schaltjahrregel)
  EVALUATE WS-MONAT
    WHEN 2
       IF (FUNCTION MOD(WS-JAHR 400) = 0)
       OR (FUNCTION MOD(WS-JAHR 4) = 0
           AND FUNCTION MOD(WS-JAHR 100) NOT = 0)
          MOVE 29 TO WS-TAG-MAX
       ELSE
          MOVE 28 TO WS-TAG-MAX
       END-IF
    WHEN 4  MOVE 30 TO WS-TAG-MAX
    WHEN 6  MOVE 30 TO WS-TAG-MAX
    WHEN 9  MOVE 30 TO WS-TAG-MAX
    WHEN 11 MOVE 30 TO WS-TAG-MAX
    WHEN OTHER MOVE 31 TO WS-TAG-MAX
  END-EVALUATE.

  IF WS-TAG < 1 OR WS-TAG > WS-TAG-MAX
      *> Unmoegliches Datum
  END-IF.
\end{lstlisting}
\end{cobolCodeSection}

\begin{cobolCodeSection}{teal!60}{Fehlerlog schreiben (STRING → WRITE)}
\begin{lstlisting}[language=Cobol]
  MOVE SPACES TO FEHLER-RECORD.

  STRING
      FUNCTION TRIM(LINE-TRIM)
      " ; Fehler: " FUNCTION TRIM(ARG-FELD)
      " - "         FUNCTION TRIM(ARG-MSG)
      INTO FEHLER-RECORD
  END-STRING.

  WRITE FEHLER-RECORD.
\end{lstlisting}
\end{cobolCodeSection}

\begin{cobolCodeSection}{teal!60}{Fehlerzähler erhöhen}
\begin{lstlisting}[language=Cobol]
  ADD 1 TO CNT-ERROR.
\end{lstlisting}
\end{cobolCodeSection}
