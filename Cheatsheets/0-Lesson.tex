\begin{center}
{\LARGE\bfseries COBOL Cheat Sheet}\\[0.5em]
{\large Cobol Grundsyntax}
\end{center}

\vspace{1em}

\begin{cobolTableSection}{green!60}{COBOL Datentypen}
\centering
\begin{tabular}{|l|l|l|l|}
\hline
\textbf{Typ} & \textbf{Format} & \textbf{Beispiel} & \textbf{Beschreibung} \\
\hline
Alphanumerisch & PIC X(n) & 01 NAME PIC X(30). & Beliebige Zeichen \\
Alphabetisch & PIC A(n) & 01 LAND PIC A(20). & Nur Buchstaben \\
Numerisch & PIC 9(n) & 01 ALTER PIC 9(3). & Nur Ziffern \\
Dezimal & PIC 9(n)V9(m) & 01 PREIS PIC 9(5)V99. & Virtueller Dezimalpunkt \\
Signiert & PIC S9(n) & 01 SALDO PIC S9(7)V99. & Mit Vorzeichen +/- \\
\hline
\end{tabular}
\end{cobolTableSection}

\begin{cobolCodeSection}{green!60}{Variablen}
\begin{lstlisting}[language=Cobol] 
    01 CUSTOMER-DATA.
        05 CUSTOMER-NAME PIC X(30) VALUE SPACES.
        05 CUSTOMER-AGE PIC 9(3) VALUE 0.
\end{lstlisting}
01 ist die Hauptebene. 05, 10, 15 die Unterebenen.
\begin{lstlisting}[language=Cobol] 
    MOVE "Emma" TO CUSTOMER-NAME. oder MOVE 25 TO CUSTOMER-AGE.
\end{lstlisting}
Zuweisung von Werten zu Variablen.
\end{cobolCodeSection}

\begin{cobolTableSection}{green!60}{Ablaufsteuerung}
\centering
\begin{tabular}{|l|l|l|}
\hline
\textbf{Statement} & \textbf{Bedeutung} & \textbf{Beispiel} \\
\hline
DISPLAY & Ausgabe & DISPLAY "Eingabe (JA/NEIN):". \\
\hline
ACCEPT & Eingabe & ACCEPT ANTWORT. \\
\hline
MOVE & Zuweisung & MOVE "AKTIV" TO STATUS. \\
\hline
IF & Bedingung & IF ANTWORT = "JA" \dots ELSE \dots END-IF. \\
\hline
PERFORM & Paragraph aufrufen & PERFORM CALC-ABSCHNITT. \\
\hline
STOP RUN & Programm beenden & STOP RUN. \\
\hline
\end{tabular}
\end{cobolTableSection}


\begin{cobolCodeSection}{green!60}{Methoden}
\begin{lstlisting}[language=Cobol] 
    PROCEDURE DIVISION.
        BEGIN.
            DISPLAY "Programm gestartet.".
            PERFORM CALC-SUM.
        STOP RUN.

        CALC-SUM.
            MOVE 100 TO SUMME.
            DISPLAY "Summe: " SUMME.
\end{lstlisting}
\end{cobolCodeSection}



\begin{cobolTableSection}{green!60}{Cobol Funktionen}
\centering
\begin{tabular}{|l|l|p{7cm}|}
\hline
\textbf{Funktion} & \textbf{Beschreibung} & \textbf{Beispiel} \\
\hline
CURRENT-DATE & Aktuelles Datum/Uhrzeit & DISPLAY FUNCTION CURRENT-DATE. \\
\hline
LENGTH(var) & Länge eines Feldes & MOVE FUNCTION LENGTH(NAME) TO LEN. \\
\hline
NUMVAL(str) & String zu Zahl & MOVE FUNCTION NUMVAL("123") TO NUM. \\
\hline
UPPER-CASE(str) & Großbuchstaben & MOVE FUNCTION UPPER-CASE(TEXT) TO UPPER. \\
\hline
LOWER-CASE(str) & Kleinbuchstaben & MOVE FUNCTION LOWER-CASE(TEXT) TO LOWER. \\
\hline
\end{tabular}
\end{cobolTableSection}


\begin{cobolCodeSection}{green!60}{Vollständiges Beispiel}
\begin{lstlisting}[language=Cobol] 
    IDENTIFICATION DIVISION.
    PROGRAM-ID. WORKSHOP-DEMO.
    ENVIRONMENT DIVISION.
    CONFIGURATION SECTION.
    SOURCE-COMPUTER. TECHNISCHE-HOCHSCHULE-MANNHEIM.
    OBJECT-COMPUTER. TECHNISCHE-HOCHSCHULE-MANNHEIM.
    DATA DIVISION.
    WORKING-STORAGE SECTION.
        01 FRAGE PIC X(30) VALUE "Ist der Kurs vorbei? (JA/NEIN)".
        01 ANTWORT PIC X(10).
        01 KURSSTATUS PIC X(15) VALUE "AKTIV".
    PROCEDURE DIVISION.
    BEGIN.
        DISPLAY FRAGE.
        ACCEPT ANTWORT.
        IF ANTWORT = "JA"
            MOVE "FEIERABEND" TO KURSSTATUS
        DISPLAY "Status: " KURSSTATUS "."
        ELSE
            DISPLAY "Der Kurs geht weiter."
        END-IF.
        STOP RUN
\end{lstlisting}
\end{cobolCodeSection}
