\begin{center}
{\LARGE\bfseries COBOL Cheat Sheet}\\[0.5em]
{\large Hilfestellungen für Aufgabe 3}
\end{center}

\vspace{1em}

\begin{cobolCodeSection}{cyan!60}{Arrays in Cobol}
\begin{lstlisting}[language=Cobol] 
  01 FirstArray
    05 ITEM OCCURS 10 TIMES PIC x VALUE "0"
\end{lstlisting}
So wird ein Array der Länge 10 mit String Variablen definiert.
Jedes Element hat als Defaultwert "0".
\begin{lstlisting}[language=Cobol] 
  ITEM(3)
\end{lstlisting}
So wird auf das dritte Element eines Array zugegriffen.
\end{cobolCodeSection}

\begin{cobolCodeSection}{cyan!60}{Consolen Eingaben in Cobol}
\begin{lstlisting}[language=Cobol] 
  ACCEPT EINGABE.
\end{lstlisting}
Mit \textit{ACCEPT} wird die Eingabe des Users in die Variable EINGABE geschrieben.
\end{cobolCodeSection}

\begin{cobolCodeSection}{cyan!60}{Systemcalls in Cobol}
\begin{lstlisting}[language=Cobol] 
  CALL "SYSTEM" USING "cat 'hello world'"
\end{lstlisting}
Mit \textit{CALL "SYSTEM"} können direkt im Programm Komandozeilen Befehle aufgerufen werden
\end{cobolCodeSection}

\begin{cobolCodeSection}{cyan!60}{Floating Points in Cobol}
\begin{lstlisting}[language=Cobol] 
  01 FLOATING_POINT PIC 9V9(4)
\end{lstlisting}
Definiert eine Kommzahl mit vier Nachkommastellen
\end{cobolCodeSection}

\begin{cobolCodeSection}{cyan!60}{Zufallszahlen in Cobol}
\begin{lstlisting}[language=Cobol] 
  COMPUTE RANDOM_NUMBER = FUNCTION RANDOM
\end{lstlisting}
Weist der Variable \textit{RANDOM\_NUMBER} eine zufällige Zahl zwischen 0.0 und 1.0 zu.
\end{cobolCodeSection}

