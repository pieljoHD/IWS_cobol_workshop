\documentclass[a4paper,11pt]{article}
\usepackage{tcolorbox}
\usepackage{listings}
\tcbuselibrary{listings,breakable}
\usepackage[margin=1.5cm]{geometry}
\usepackage{xcolor}
\usepackage[T1]{fontenc}
\usepackage[utf8]{inputenc}
\setlength\parindent{0pt}
\setlength{\parskip}{0.8em}
\setlength{\fboxsep}{6pt}

\definecolor{hellblau}{RGB}{173,216,230}

\newtcolorbox{theorieSection}[2]{
  colframe=#1,
  colback=white,
  coltitle=#1,
  fonttitle=\bfseries,
  title=#2
}

\newtcolorbox{cobolCodeSection}[2]{
  colframe=#1,
  colback=white,
  coltitle=white,
  fonttitle=\bfseries,
  title=#2,
  listing only,
  listing options={
    language=Cobol,
    basicstyle=\ttfamily\small,
    numbers=left,
    numberstyle=\tiny\color{gray},
    tabsize=2,
    breaklines=true
  }
}


% color definitions for Cobol
\lstdefinelanguage{Cobol}{
  keywords={
    FILE-CONTROL, SELECT, ASSIGN, ORGANIZATION IS, PIC, COMPUTE
    ADD,SUBTRACT, MULTIPLY, DIVIDE, GIVING
    },
  sensitive=false,
  comment=[l]{*>}
}

\lstset{
  language=Cobol,
  basicstyle=\ttfamily\small,
  keywordstyle=\color{blue}\bfseries,
  commentstyle=\color{green}\sffamily\small\itshape,
  stringstyle=\color{orange},
  numbers=left,
  numberstyle=\tiny\color{gray},
  stepnumber=1,
  numbersep=5pt,
  showspaces=false,
  showstringspaces=false,
  tabsize=2,
  breaklines=true
}


\begin{document}

\begin{center}
{\LARGE\bfseries COBOL Cheat Sheet}\\[0.5em]
{\large Hilfestellungen für Aufgabe 1}
\end{center}

\vspace{1em}

\begin{cobolCodeSection}{cyan!60}{Datei definieren}
\begin{lstlisting}[language=Cobol] 
  FILE-CONTROL. 
    SELECT VARIABLENNAME
      ASSIGN TO "DATEINAME"
      ORGANIZATION IS EINLESEART. 

    FILE SECTION.
      FD VARIABLENNAME.
        01 BUCHUNG-LINE PIC X(80).
\end{lstlisting}
\end{cobolCodeSection}

\begin{cobolCodeSection}{cyan!60}{Datei öffnen und schlie\ss en}
\begin{lstlisting}[language=Cobol] 
  OPEN INPUT VARIABLENNAME.
  CLOSE BUCHUNGEN.
\end{lstlisting}
\end{cobolCodeSection}


\begin{cobolCodeSection}{cyan!60}{Eine Zeile einlesen}
\begin{lstlisting}[language=Cobol] 
  READ BUCHUNGEN
  END-READ
\end{lstlisting}
\end{cobolCodeSection}


\begin{cobolCodeSection}{cyan!60}{Berechnungen}
\begin{lstlisting}[language=Cobol] 
  COMPUTE VARIABLENNAME = X + Y * Z.
\end{lstlisting}
\end{cobolCodeSection}

\begin{cobolCodeSection}{cyan!60}{Mathematische Funktionen}
\begin{lstlisting}[language=Cobol] 
  ADD X TO Y GIVING SUMXY.
  SUBTRACT Y FROM X GIVING SUBTRACTION.
  MULTIPLY X BY Y GIVING PRODUCT.
  DIVIDE X BY Y GIVING QUOTIENT.
\end{lstlisting}
\end{cobolCodeSection}

\newpage
\begin{center}
{\LARGE\bfseries COBOL Cheat Sheet}\\[0.5em]
{\large Hilfestellungen für Aufgabe 3}
\end{center}

\vspace{1em}

\begin{cobolCodeSection}{cyan!60}{Arrays in Cobol}
\begin{lstlisting}[language=Cobol] 
  01 FirstArray
    05 ITEM OCCURS 10 TIMES PIC x VALUE "0"
\end{lstlisting}
So wird ein Array der Länge 10 mit String Variablen definiert.
Jedes Element hat als Defaultwert "0".
\begin{lstlisting}[language=Cobol] 
  ITEM(3)
\end{lstlisting}
So wird auf das dritte Element eines Array zugegriffen.
\end{cobolCodeSection}

\begin{cobolCodeSection}{cyan!60}{Consolen Eingaben in Cobol}
\begin{lstlisting}[language=Cobol] 
  ACCEPT EINGABE.
\end{lstlisting}
Mit \textit{ACCEPT} wird die Eingabe des Users in die Variable EINGABE geschrieben.
\end{cobolCodeSection}

\begin{cobolCodeSection}{cyan!60}{Systemcalls in Cobol}
\begin{lstlisting}[language=Cobol] 
  CALL "SYSTEM" USING "cat 'hello world'"
\end{lstlisting}
Mit \textit{CALL "SYSTEM"} können direkt im Programm Komandozeilen Befehle aufgerufen werden
\end{cobolCodeSection}

\begin{cobolCodeSection}{cyan!60}{Floating Points in Cobol}
\begin{lstlisting}[language=Cobol] 
  01 FLOATING_POINT PIC 9V9(4)
\end{lstlisting}
Definiert eine Kommzahl mit vier Nachkommastellen
\end{cobolCodeSection}

\begin{cobolCodeSection}{cyan!60}{Zufallszahlen in Cobol}
\begin{lstlisting}[language=Cobol] 
  COMPUTE RANDOM_NUMBER = FUNCTION RANDOM
\end{lstlisting}
Weist der Variable \textit{RANDOM\_NUMBER} eine zufällige Zahl zwischen 0.0 und 1.0 zu.
\end{cobolCodeSection}



\end{document}
